\documentclass[10pt, letterpaper, titlepage]{report}

\title{PhD Thesis}
\author{Michael Dub\'e}
\date{August 2022}

\bibliographystyle{plain}


\begin{document}

\maketitle

\begin{abstract}
    Abstract.
\end{abstract}

\tableofcontents
\clearpage

\chapter{August 2022}
\section{Week 4}
\subsection{Tuesday, August 23$^{rd}$, 2022}
    \begin{itemize}
        \item I started the GitHub repository, Overleaf Project, and Research Log for my thesis/dissertation.
        \item Had a meeting with Steffen covering my proposal, some questions I had about bioinformatics tools, and the biological concepts related to dehydrins.
        \item Met with Joey to discuss Dan and Side-Effect Machines (SEMs).  Read the paper he sent me on them before our next meeting.  Also, contact Giulia from the library at Brock regarding Public Speaking Training.
        \item Filled the References with souces from my Proposal.
        \item Created the README.md file.
        \item Learned the MEME \cite{MEME} and the related MEME Suite generate both the weighted matrix for amino acids at each index as well as determine the likelihood a given sequence matches the sequence in a provided weighted matrix.
        \item I will be re-constructing the database of dehydrin sequences myself in order to have the experience and as a check to see how mine compares to Riley's and other's work.
        \item Learned that most sequences are sourced from reverse transcription of RNA rather than mass spectrometry.
        \item Read \cite{GnomeAnal2017, Riley2019Feb} for our meeting tomorrow.
        \item Other tools include: PsiBlast, etc.
        \item Qualifying Exam (QE) is an updated research proposal and report on the work already completed towards the research questions.  Aiming for Winter 2023, could be end of Fall 2022.
    \end{itemize}

\subsection{Wednesday, August 24$^{th}$, 2022}
    \begin{itemize}
        \item Had a meeting with Steffen and Sheridan to discuss my Research Proposal and next steps for my research.
        \item First I will read \cite{GnomeAnal2017, Riley2019Feb} and use the databases and methods described to generate my own set of amino acid sequences and motifs.  Then I will use that to start some work with the basesprayer.  This could be evolving them to match a single sequence or particular motifs (K-segments).
        \item I will use Steffen's group meetings to learn some biological concepts, teaching them to the group.  Steffen will compile a list of topics and a suggested textbook.
        \item My dissertation will include my previous work using bitsprayers as a network representation and for procedural content generation.
        \item Possible courses: BINF*6970, CIS*6060, and MCB*6370.  All are offered in the winter.
        \item Ask Rachel for advice on my QE; Sheridan and Steffen are not able to help directly.
        \item Think about implementing basesprayers/SEMs for amino acid sequences.
    \end{itemize}

\subsection{Thursday, August 25$^{th}$, 2022}
    \begin{itemize}
        \item 
    \end{itemize}

\bibliography{References}

\end{document}